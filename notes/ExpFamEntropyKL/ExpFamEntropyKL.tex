\documentclass[a4paper, 11pt]{article}

\usepackage{amsmath, amssymb, hyperref, ../../vimacros}
\hypersetup{breaklinks=true, colorlinks=true, linkcolor=blue, urlcolor=blue, citecolor=blue}

\author{Philip Schulz and Wilker Aziz}
\title{Exponential Family Entropy and KL divergence}
\date{last modified: \today}

\begin{document}

\maketitle

\begin{abstract}
This note shows how to compute the entropy and KL divergence (and by extension many other information-theoretic quantities) of 
exponential family distributions. We use the univariate Gaussian distribution as a running example and encourage the reader to
apply the learned techniques to distributions in exercises. We expect the reader to be familiar with exponential families 
and some of their properties (such as moment computation).
\end{abstract}

\section{Exponential Family Form}

Most exponential family distributions are standardly written in what is called their canonical form. 
In order to apply results proved for 
exponential families to them, it is often convenient to rewrite them in their natural parametrisation. The general way of doing
this is to apply the identity function $ \exp(\log(\cdot)) $ to them and reorder terms. Before we do so for the Gaussian, let us
introduce the notation we will use for exponential families throughout.
\begin{equation} \label{eq:expFam}
p(x|\overbrace{\theta}^{\substack{\text{canonical} \\ \text{parameters}}}) = \overbrace{h(x)}^{\substack{\text{base} \\ \text{measure}}} \exp\left( \underbrace{\eta(\theta)^{\top}}_{\substack{\text{natural} \\ \text{parameters}}} \times \overbrace{t(x)}^{\substack{\text{sufficient} \\ \text{statistics}}}  -\underbrace{A(\eta(\theta))}_{\text{log-normalizer}} \right)
\end{equation}
By default we assume that all terms above are vector-valued. Notice that the canonical parameters may be a function of any conditioning
context and thus Equation~\eqref{eq:expFam} covers marginal as well as conditional distributions.

\paragraph{Gaussian Distribution} We first write the Gaussian distribution with mean $ \mu $ and variance $ \sigma^{2} $ in its 
canonical form. In that case we have $ \theta^{\top} = \begin{bmatrix} \mu & \sigma^{2} \end{bmatrix} $.
\begin{equation}\label{eq:canonicalGaussian}
p(x|\mu, \sigma^{2}) = \frac{1}{\sqrt{2\pi}\sigma} \exp \left(-\frac{1}{2}\left(\frac{x - \mu}{\sigma}\right)^{2}\right)
\end{equation}

Now we apply the identity function to transform this density into its exponential family form.
\begin{subequations}
\begin{align}
p(x|\mu, \sigma^{2}) &= \exp \left( \log \left( \frac{1}{\sqrt{2\pi}\sigma} \exp \left(-\frac{1}{2}\left(\frac{x - \mu}{\sigma}\right)^{2}\right) \right) \right) \\
&= \exp \left( \log \left( \frac{1}{\sqrt{2\pi}} \right) - \log \sigma -\frac{1}{2}\frac{(x - \mu)^{2}}{\sigma^{2}}\right) \\
&= \frac{1}{\sqrt{2\pi}}\exp \left( - \log \sigma -\frac{x^{2} - 2x\mu + \mu^{2}}{2\sigma^{2}}\right) \\
&= \frac{1}{\sqrt{2\pi}} \exp \left(\frac{2x\mu -x^{2}}{2\sigma^{2}} - \log\sigma - \frac{\mu^{2}}{2\sigma^{2}}\right) \\
&= \frac{1}{\sqrt{2\pi}} \exp \left( 
\begin{bmatrix} \frac{\mu}{\sigma^{2}} & -\frac{1}{2\sigma^{2}} \end{bmatrix} \times 
\begin{bmatrix} x \\ x^{2} \end{bmatrix}
- \log\sigma - \frac{\mu^{2}}{2\sigma^{2}}\right)
\end{align}
\end{subequations}
We recognise this as the exponential family form of the Gaussian with
\begin{equation}
\begin{aligned}
h(x) = \frac{1}{\sqrt{2\pi}} && t(x) &= \begin{bmatrix} x \\ x^{2} \end{bmatrix} \\
A(\eta(\theta)) = \log \sigma + \frac{\mu^{2}}{2\sigma^{2}} && \eta(\theta) &= \begin{bmatrix} \frac{\mu}{\sigma^{2}} \\ -\frac{1}{2\sigma^{2}} \end{bmatrix} \ .
\end{aligned}
\end{equation}

\paragraph{Exercise} Derive the exponential family form of the exponential and binomial distribution.

\section{Entropy of Exponential Families}

First, recall the definition of entropy:
\begin{equation}
\Ent{X} = -\E{\log(X)}
\end{equation}
A convenient property of exponential families is that the only terms depending on $ X $ are the base measure and the sufficient
statistics. This means we can split up the expectation term.
\begin{subequations}
\begin{align}
\Ent{X} &= -\E{\log(h(X)) + \eta(\theta))^{\top} t(X) - A(\eta(\theta))} \\
&=- \E{\log(h(X))} + \eta(\theta))^{\top} \E{t(X)} - A(\eta(\theta))
\end{align}
\end{subequations}
For most distributions the expectation of the log base measure is a constant because $ h(\cdot) $ does not actually depend on $ X $ (see the Gaussian example above). In order to compute the expected sufficient statistics we exploit another convenient property
of exponential families.
\begin{equation}
\E{t(X)} = \frac{d}{d\eta(\theta))}A(\eta(\theta))
\end{equation}

%\paragraph{Gaussian Entropy} We now compute the entropy of the Gaussian using the above properties. First, we compute the vector
%of partial derivatives to get the expected sufficient statistics.
%\begin{align}
%\frac{\partial}{\partial \frac{\mu}{\sigma^{2}}} \left\{ \log \sigma + \frac{\mu^{2}}{2\sigma^{2}} \right\} = \frac{\mu}{2}
%\end{align}

\end{document}